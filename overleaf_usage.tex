% 在 Overleaf 中使用双栏图片的示例代码

% 方式1: 使用 figure* 环境(跨双栏,占满整个页面宽度)
% 适用于图片宽度为双栏宽度的情况
\begin{figure*}[t]
    \centering
    \includegraphics[width=\textwidth]{electra_arch.png}
    \caption{ELECTRA-small architecture for NLI. The model processes the concatenated premise and hypothesis through the ELECTRA-small backbone (12 transformer layers), extracts the \texttt{[CLS]} token representation, and applies a classification head to produce the final prediction.}
    \label{fig:electra_arch}
\end{figure*}

% 方式2: 如果使用 figure 环境(单栏),图片会自动缩放
% 适用于图片放在单栏中的情况
\begin{figure}[t]
    \centering
    \includegraphics[width=\columnwidth]{electra_arch.png}
    \caption{ELECTRA-small architecture for NLI.}
    \label{fig:electra_arch}
\end{figure}

% 方式3: 精确控制宽度(推荐用于双栏图片)
\begin{figure*}[t]
    \centering
    \includegraphics[width=0.95\textwidth]{electra_arch.png}
    \caption{ELECTRA-small architecture for NLI. The model processes the concatenated premise and hypothesis through the ELECTRA-small backbone (12 transformer layers), extracts the \texttt{[CLS]} token representation, and applies a classification head to produce the final prediction.}
    \label{fig:electra_arch}
\end{figure*}

% 方式4: 使用 scale 参数(保持原始比例)
\begin{figure*}[t]
    \centering
    \includegraphics[scale=0.8]{electra_arch.png}
    \caption{ELECTRA-small architecture for NLI.}
    \label{fig:electra_arch}
\end{figure*}

